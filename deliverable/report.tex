% Options for packages loaded elsewhere
\PassOptionsToPackage{unicode}{hyperref}
\PassOptionsToPackage{hyphens}{url}
\PassOptionsToPackage{dvipsnames,svgnames,x11names}{xcolor}
%
\documentclass[
  reportpaper,
  DIV=11,
  numbers=noendperiod]{scrartcl}

\usepackage{amsmath,amssymb}
\usepackage{iftex}
\ifPDFTeX
  \usepackage[T1]{fontenc}
  \usepackage[utf8]{inputenc}
  \usepackage{textcomp} % provide euro and other symbols
\else % if luatex or xetex
  \usepackage{unicode-math}
  \defaultfontfeatures{Scale=MatchLowercase}
  \defaultfontfeatures[\rmfamily]{Ligatures=TeX,Scale=1}
\fi
\usepackage{lmodern}
\ifPDFTeX\else  
    % xetex/luatex font selection
\fi
% Use upquote if available, for straight quotes in verbatim environments
\IfFileExists{upquote.sty}{\usepackage{upquote}}{}
\IfFileExists{microtype.sty}{% use microtype if available
  \usepackage[]{microtype}
  \UseMicrotypeSet[protrusion]{basicmath} % disable protrusion for tt fonts
}{}
\makeatletter
\@ifundefined{KOMAClassName}{% if non-KOMA class
  \IfFileExists{parskip.sty}{%
    \usepackage{parskip}
  }{% else
    \setlength{\parindent}{0pt}
    \setlength{\parskip}{6pt plus 2pt minus 1pt}}
}{% if KOMA class
  \KOMAoptions{parskip=half}}
\makeatother
\usepackage{xcolor}
\setlength{\emergencystretch}{3em} % prevent overfull lines
\setcounter{secnumdepth}{-\maxdimen} % remove section numbering
% Make \paragraph and \subparagraph free-standing
\ifx\paragraph\undefined\else
  \let\oldparagraph\paragraph
  \renewcommand{\paragraph}[1]{\oldparagraph{#1}\mbox{}}
\fi
\ifx\subparagraph\undefined\else
  \let\oldsubparagraph\subparagraph
  \renewcommand{\subparagraph}[1]{\oldsubparagraph{#1}\mbox{}}
\fi


\providecommand{\tightlist}{%
  \setlength{\itemsep}{0pt}\setlength{\parskip}{0pt}}\usepackage{longtable,booktabs,array}
\usepackage{calc} % for calculating minipage widths
% Correct order of tables after \paragraph or \subparagraph
\usepackage{etoolbox}
\makeatletter
\patchcmd\longtable{\par}{\if@noskipsec\mbox{}\fi\par}{}{}
\makeatother
% Allow footnotes in longtable head/foot
\IfFileExists{footnotehyper.sty}{\usepackage{footnotehyper}}{\usepackage{footnote}}
\makesavenoteenv{longtable}
\usepackage{graphicx}
\makeatletter
\def\maxwidth{\ifdim\Gin@nat@width>\linewidth\linewidth\else\Gin@nat@width\fi}
\def\maxheight{\ifdim\Gin@nat@height>\textheight\textheight\else\Gin@nat@height\fi}
\makeatother
% Scale images if necessary, so that they will not overflow the page
% margins by default, and it is still possible to overwrite the defaults
% using explicit options in \includegraphics[width, height, ...]{}
\setkeys{Gin}{width=\maxwidth,height=\maxheight,keepaspectratio}
% Set default figure placement to htbp
\makeatletter
\def\fps@figure{htbp}
\makeatother

\usepackage{fancyhdr}
\pagestyle{fancy}
\fancyhead{}
\fancyhead[C]{Technical report}
\fancyfoot[L]{Centre for Data Science}
\fancyfoot[R]{\today}
\renewcommand{\footrulewidth}{0.4pt} % Adds a horizontal line in the footer
\KOMAoption{captions}{tableheading}
\makeatletter
\@ifpackageloaded{caption}{}{\usepackage{caption}}
\AtBeginDocument{%
\ifdefined\contentsname
  \renewcommand*\contentsname{Table of contents}
\else
  \newcommand\contentsname{Table of contents}
\fi
\ifdefined\listfigurename
  \renewcommand*\listfigurename{List of Figures}
\else
  \newcommand\listfigurename{List of Figures}
\fi
\ifdefined\listtablename
  \renewcommand*\listtablename{List of Tables}
\else
  \newcommand\listtablename{List of Tables}
\fi
\ifdefined\figurename
  \renewcommand*\figurename{Figure}
\else
  \newcommand\figurename{Figure}
\fi
\ifdefined\tablename
  \renewcommand*\tablename{Table}
\else
  \newcommand\tablename{Table}
\fi
}
\@ifpackageloaded{float}{}{\usepackage{float}}
\floatstyle{ruled}
\@ifundefined{c@chapter}{\newfloat{codelisting}{h}{lop}}{\newfloat{codelisting}{h}{lop}[chapter]}
\floatname{codelisting}{Listing}
\newcommand*\listoflistings{\listof{codelisting}{List of Listings}}
\makeatother
\makeatletter
\makeatother
\makeatletter
\@ifpackageloaded{caption}{}{\usepackage{caption}}
\@ifpackageloaded{subcaption}{}{\usepackage{subcaption}}
\makeatother
\ifLuaTeX
  \usepackage{selnolig}  % disable illegal ligatures
\fi
\usepackage{bookmark}

\IfFileExists{xurl.sty}{\usepackage{xurl}}{} % add URL line breaks if available
\urlstyle{same} % disable monospaced font for URLs
\hypersetup{
  pdftitle={Technical report},
  pdfauthor={James Hogg, Kerrie Mengersen},
  colorlinks=true,
  linkcolor={blue},
  filecolor={Maroon},
  citecolor={Blue},
  urlcolor={Blue},
  pdfcreator={LaTeX via pandoc}}

\title{Technical report}
\usepackage{etoolbox}
\makeatletter
\providecommand{\subtitle}[1]{% add subtitle to \maketitle
  \apptocmd{\@title}{\par {\large #1 \par}}{}{}
}
\makeatother
\subtitle{Email Efficiency Calculators}
\author{James Hogg, Kerrie Mengersen}
\date{April 4, 2024}

\begin{document}
\maketitle

\begin{center}
\includegraphics[width=0.3\textwidth]{../ed/static/ed/CDS-Logo-Blue.png}
\end{center}

\newpage{}

\section{Standard Calculator}\label{standard-calculator}

The following notation is used:

\begin{itemize}
\tightlist
\item
  \(N_i\): Number of emails of type \(i\), where \(i \in \{1,2,3\}\)
  corresponds to the three email types.
\item
  \(R_i\): Time to read each email of type \(i\) in minutes.
\item
  \(A_i\): Time to take action on each email of type \(i\) in minutes.
\item
  \(W_i\): Time to write or respond to each email of type \(i\) in
  minutes.
\item
  \(P_i\): Probability that an email of type \(i\) is essential.
\item
  \(T_{total}\): Total email time.
\item
  \(T_{essential}\): Total email time that is essential.
\item
  \(T_{nonessential}\): Total email time that is nonessential.
\end{itemize}

The total email time per type is calculated by summing the product of
the number of emails and the respective email activity times (e.g.,
read, action, responding). The formula for the total email time for
email type \(i\) is

\[ T_i = N_i \times (R_i + A_i + W_i) \].

Note that the values in the right pie chart are \(T_1,T_2,T_3\). The
total email time in minutes is derived by summing.

\[ T_{total} = \sum_{i=1}^{3} T_i \]

To differentiate between essential and nonessential email time, we use
the probability \(P_i\).

\[ T_{essential,i} = P_i \times T_i \]

\[ T_{nonessential,i} = (1 - P_i) \times T_i \]

Then, to find the email time spent on essential and nonessential emails
across all email types, we sum across the respective times for each
email type:

\[ T_{essential} = \sum_{i=1}^{3} T_{essential,i} \]

\[ T_{nonessential} = \sum_{i=1}^{3} T_{nonessential,i} \]

The values in the left pie chart are \(T_{essential}\) and
\(T_{nonessential}\).

The cost associated with the managing emails is determined by
multiplying the email time, in hours, by the pay rate per hour. The cost
of time spent on all emails and the cost of time spent on essential and
nonessential emails is calculated as follows:

\[ \text{Cost}_{total} = ( T_{total} \times \text{Pay} ) / 60\]

\[ \text{Cost}_{essential} = ( T_{essential} \times \text{Pay}  ) / 60\]

\[ \text{Cost}_{nonessential} = ( T_{nonessential} \times \text{Pay}  ) / 60\]

\newpage{}

\section{Advanced Calculator}\label{advanced-calculator}

The advanced calculator introduces variability into the derivations,
both in the number of received emails and the time taken for the email
activities (e.g., read, action, respond). The time for the email
activities are modeled using the Gamma distribution with its parameters
derived from the user inputs. We use similar notation as above, with the
introduction of the following new terms.

\begin{itemize}
\tightlist
\item
  \(N_{lower,i}, N_{upper,i}\): User-specified lower and upper bounds
  for the number of emails recieved of email type \(i\)
\item
  \(R_{lower,i}, R_{upper,i}, A_{lower,i}, A_{upper,i}, W_{lower,i}, W_{upper,i}\):
  User-specified lower and upper bounds for reading, action, and writing
  times for email type \(i\), respectively.
\end{itemize}

\begin{enumerate}
\def\labelenumi{\arabic{enumi}.}
\item
  \textbf{Sample the number of emails}:
  \[ N_i \sim \text{Uniform}(N_{lower,i}, N_{upper,i}) \]
\item
  \textbf{Parameterisation of distribution of time to complete email
  activities}:

  \begin{itemize}
  \tightlist
  \item
    The parameters for the Gamma distribution are derived from the
    user-specified lower and upper bounds.
  \item
    Consider reading times as an example.
    \[ \mu_{reading,i} = \frac{R_{upper,i} + R_{lower,i}}{2} \]
    \[ \sigma_{reading,i}^2 = \left(\frac{R_{upper,i} - R_{lower,i}}{3.92}\right)^2 \]
  \item
    Similar calculations are made for action and writing times.
  \end{itemize}
\item
  \textbf{Simulate email activity times}:

  \begin{itemize}
  \tightlist
  \item
    Consider reading times for email type \(i\). Simulate a vector of
    length \(N_i\) from the following:
    \[\text{Gamma}(k = \frac{\mu_{reading,i}^2}{\sigma_{reading,i}^2}, \theta = \frac{\sigma_{reading,i}^2}{\mu_{reading,i}}) \]
  \item
    Complete the above simulation for each email type and email activity
  \end{itemize}
\item
  \textbf{Calculate total email time per email type}:
\end{enumerate}

\begin{itemize}
\tightlist
\item
  The total email time, \(T_i\), is the sum of the sampled vectors for
  each email activity.
\end{itemize}

\begin{enumerate}
\def\labelenumi{\arabic{enumi}.}
\setcounter{enumi}{4}
\tightlist
\item
  \textbf{Stratify by essential and nonessential emails}:

  \begin{itemize}
  \tightlist
  \item
    The distinction between essential and nonessential time is made by
    randomly assigning each of the simulated emails in the previous step
    a 1 (denoting essential) or a zero (denoting nonessential).
  \item
    Simulate \(N_i\)-dimensional vectors, \(\boldsymbol{G}_i\), of
    assignments using a Bernoulli trail with probability \(P_i\)
  \item
    Derive the esssential and nonessential email time using the dot
    product of \(\boldsymbol{G}_i\) and the vector of total email time
    for each email.
  \end{itemize}
\item
  \textbf{Calculate total time}:

  \begin{itemize}
  \tightlist
  \item
    The total email time is \(T_1 + T_2 + T_3\).
  \end{itemize}
\item
  \textbf{Calculate yearly hours and cost}

  \begin{itemize}
  \tightlist
  \item
    The formula used to derive the yearly hours and cost values follow
    those given in the standard calculator.
  \end{itemize}
\end{enumerate}

By running the above steps 1000 times, the advanced calculator estimates
the distribution of the output values, including total, essential, and
nonessential times, and yearly hours and cost. To summarise the
simulations a point estimate (e.g., median) and interval (2.5\% and
97.5\% quantiles) are reported in the dashboard. The pie charts in the
advanced calculator are the point estimates only.



\end{document}
